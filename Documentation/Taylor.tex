\documentclass[titlepage=false,12pt]{article}
\usepackage {amsmath}
\usepackage{amssymb}
\pdfoutput = 1 
\usepackage {graphicx}
\newcommand {\bxi}{\mbox{\boldmath$\xi$}}
\allowdisplaybreaks

\begin{document}

\title{Simplified Forced Magnetic Reconnection}
\date{}
\maketitle

\section{Simplified Analytic Calculation}
\subsection{Analysis}
Let $\hat{t}= t/(S^{1/3}\,\tau_H)$, where $t$ represents time, $S$ is the Lundquist number, and $\tau_H$
the hydromagnetic timescale. Let ${\mit\Psi}_b(\hat{t})$ be the driving flux, and ${\mit\Psi}_0(\hat{t})$ the
reconnected flux. Let $E_{sb}$ be a (real) dimensionless coupling constant, and $E_{ss}$ the  (real) dimesionless
tearing stability index.

Let
\begin{align}
\bar{\mit\Psi}_b(g) = \int_0^\infty {\mit\Psi}_b(\hat{t})\,{\rm e}^{-g\,\hat{t}}\,d\hat{t},\\[0.5ex]
\bar{\mit\Psi}_0(g) = \int_0^\infty {\mit\Psi}_0(\hat{t})\,{\rm e}^{-g\,\hat{t}}\,d\hat{t}.
\end{align}
In fact, if we assume that ${\mit\Psi}_b(\hat{t})=0$ for $\hat{t}<0$, and
\begin{equation}
{\mit\Psi}_b(\hat{t}) = {\mit\Psi}_{b\,0}
\end{equation}
for $\hat{t}\geq 0$, then 
\begin{equation}
\bar{\mit\Psi}_b(g) = \frac{{\mit\Psi}_{b\,0}}{g}.
\end{equation}

The Laplace transformed simplified layer equation is
\begin{equation}\label{e5}
\frac{d}{dp}\!\left(\frac{p^2}{Q+p^2}\,\frac{d\bar{\phi}}{dp}\right) - Q\,p^2\,\bar{\phi}=0,
\end{equation}
where
\begin{equation}
Q = g+ {\rm i}\,Q_E.
\end{equation}
Here, $Q_E=S^{1/3}\,\omega_E\,\tau_H$, where $\omega_E$ is the E-cross-B frequency.
Equation~(\ref{e5}) must be solved subject to the boundary conditions that $\bar{\phi}(p)\rightarrow 0$ 
as $p\rightarrow \infty$, and
\begin{equation}
\bar{\phi}(p) \rightarrow \frac{\hat{\mit\Delta}_s(g)}{\pi\,p }+ 1 + {\cal O}(p)
\end{equation}
as $p\rightarrow 0$. 
Furthermore,
\begin{equation}
F_s(g) = - \int_0^\infty \frac{p^2}{Q+p^2}\,\frac{d\bar{\phi}}{dp}\,dp.
\end{equation}
Finally,
\begin{equation}
\bar{\mit\Psi}_0(g) = \frac{E_{sb}\,F_s(g)\,\bar{\mit\Psi}_b(g)}{g\,[S^{1/3}\,\hat{\mit\Delta}_s(g)-E_{ss}]}.
\end{equation}

It can be demonstrated that
\begin{equation}
\frac{p^2}{Q+p^2}\,\frac{d\bar{\phi}}{dp}= \frac{Q^{1/4}\,\Gamma(a)\,\Gamma(5/2)}{3\pi}\,{\rm e}^{-z/2}\,U(a,-1/2,z),
\end{equation}
where $U(a,b,z)$ is a confluent hypergeometric function, and
\begin{align}
z &= Q^{1/2}\,p^2,\\[0.5ex]
a&= \frac{1}{4}\,(Q^{3/2}-1).
\end{align}
It follows that
\begin{equation}
F_s(g) =-\frac{\Gamma(a)\,\Gamma(5/2)}{6\pi}\int_0^\infty z^{-1/2}\,{\rm e}^{-z/2}\,U(a,-1/2,z)\,dz.
\end{equation}
However, according to G\&R 7.621.6,
\begin{equation}
\int_0^\infty z^{-1/2}\,{\rm e}^{-z/2}\,U(a,-1/2,z)\,dc = \frac{\Gamma(1/2)\,\Gamma(2)}{\Gamma(a+2)}\,F(1/2, 2;a+2;1/2),
\end{equation}
where $F(a,b;c;z)$ is a hypergeometric function. Thus,
\begin{equation}
F_s(g) = - \frac{\Gamma(a)\,\Gamma(5/2)}{6\pi} \,\frac{\Gamma(1/2)\,\Gamma(2)}{\Gamma(a+2)}\,F(1/2, 2;a+2;1/2).
\end{equation}
Finally,
\begin{equation}
\hat{\mit\Delta}_s(g) = -\frac{\pi}{8}\,\frac{\Gamma(a)\,Q^{5/4}}{\Gamma(a+3/2)}.
\end{equation}

\subsection{Non-Constant-$\psi$ Limit}
Suppose that $|Q|\gg 1$, which implies that $|a|\gg 1$.  It follows that
\begin{align}
F_s(g) &\simeq - \frac{\Gamma(a)\,\Gamma(5/2)}{6\pi}\,\frac{\Gamma(1/2)\,\Gamma(2)}{\Gamma(a+2)}\simeq-\frac{\Gamma(5/2)\,\Gamma(1/2)\,\Gamma(2)}{6\pi\,a^2}\nonumber\\[0.5ex]
&= -\frac{1}{8\,a^2}\simeq -\frac{2}{Q^{3}}.
\end{align}
Furthermore,
\begin{equation}
\hat{\mit\Delta}_s\simeq -\frac{\pi\,Q^{5/4}}{8\,a^{3/2}}\simeq-\frac{\pi}{Q}.
\end{equation}
Let us assume that $S^{1/3}/|Q|\gg 1\sim{\cal O}( E_{ss})$, because if $S^{1/3}/|Q|\ll 1$ then the layer width is not much smaller
than the plasma minor radius, which invalidates asymptotic matching. 
Thus, we find that
\begin{equation}
\bar{\mit\Psi}_0(g) = \frac{2}{\pi}\,\frac{E_{sb}\,{\mit\Psi}_{b\,0}}{S^{1/3}}\,\frac{1}{g\,(g+{\rm i}\,Q_E)^2}.
\end{equation}
If we assume that $|Q_E|\sim{\cal O}(1)$ then 
we get
\begin{equation}
\bar{\mit\Psi}_0(g) \simeq \frac{2}{\pi}\,\frac{E_{sb}\,{\mit\Psi}_{b\,0}}{S^{1/3}}\,\frac{1}{g^3},
\end{equation}
for
\begin{equation}
1\ll g\ll S^{1/3}.
\end{equation}
It follows that
\begin{equation}
{\mit\Psi}_0(\hat{t}) = \frac{E_{sb}\,{\mit\Psi}_{b\,0}}{\pi\,S^{1/3}} \,\hat{t}^{\,2},
\end{equation}
which is valid for
\begin{equation}
S^{-1/3} \ll \hat{t}\ll 1.
\end{equation}
More generally, if $|Q_E|\gg 1$ then (Erd\'{e}lyi 5.2.8)
\begin{equation}
{\mit\Psi}_0(\hat{t}) = \frac{2\,E_{sb}\,{\mit\Psi}_{b\,0}}{\pi\,S^{1/3}} \,\frac{(1+{\rm i}\,Q_E\,\hat{t})\,{\rm e}^{-{\rm i}\,Q_E\,\hat{t}}-1}{Q_E^{\,2}}.
\end{equation}

\subsection{Constant-$\psi$ Limit}
Suppose that $|Q|\ll 1$, which implies that $a=-1/4$. If we assume that $|Q_E|\ll 1$ then this limit corresponds to $|g|\gg 1$, or 
\begin{equation}
1\ll \hat{t}.
\end{equation}
It follows that 
\begin{equation}
F_s(g) = - \frac{\Gamma(-1/4)\,\Gamma(5/2)}{6\pi} \,\frac{\Gamma(1/2)\,\Gamma(2)}{\Gamma(7/4)}\,F(1/2, 2;7/4;1/2)=1.
\end{equation}
Moreover,
\begin{equation}
\hat{\mit\Delta}_s = -\frac{\pi}{8}\,\frac{\Gamma(-1/4)}{\Gamma(5/4)}\,Q^{5/4}= \frac{2\pi\,\Gamma(3/4)}{\Gamma(1/4)}\,Q^{5/4}.
\end{equation}
It follows that
\begin{equation}
\bar{\mit\Psi}_0(g) = \frac{E_{sb}\,{\mit\Psi}_{b\,0}}{g\left[c\,S^{1/3}\,(g+{\rm i}\,Q_E)^{5/4}-E_{ss}\right]},
\end{equation}
where
\begin{equation}
c =  \frac{2\pi\,\Gamma(3/4)}{\Gamma(1/4)}.
\end{equation}
Now,
\begin{equation}
{\mit\Psi}_0(\hat{t}) = \frac{1}{2\pi\,{\rm i}}\int_{-{\rm i}\,\infty}^{{\rm i}\,\infty}
\frac{E_{sb}\,{\mit\Psi}_{b\,0}\,{\rm e}^{\,g\,\hat{t}}}{g\left[c\,S^{1/3}\,(g+{\rm i}\,Q_E)^{5/4}-E_{ss}\right]}\,dg,
\end{equation}
which can be written
\begin{equation}\label{e23}
{\mit\Psi}_0(\hat{t}) = \frac{E_{sb}\,{\mit\Psi}_{b\,0}}{(-E_{ss})}\,\frac{1}{2\pi\,{\rm i}}\int_{-{\rm i}\,\infty}^{{\rm i}\,\infty}
\frac{{\rm e}^{\,(p-{\rm i}\,Q_E)\,\hat{t}}}{(p-{\rm i}\,Q_E)\,(\lambda\,p^{5/4}+1)}\,dp,
\end{equation}
where
\begin{equation}
\lambda = \frac{c\,S^{1/3}}{(-E_{ss})}.
\end{equation}
Here, we are assuming that $E_{ss}<0$. 
The contour integral in Eq.~(23) can be decomposed into elements from the three poles (at $p_E = {\rm i}\,Q_E$, and 
$p_\pm= {\rm e}^{\pm\,{\rm i}\,4\pi/5}\,\lambda^{-4/5}$), respectively) plus a contribution from a branch cut along the $p=
{\rm e}^{\,{\rm i}\,\pi}\,u$ axis (where $u$ is real and positive).
The contribution from the pole at $p=p_E$ is
\begin{equation}
\frac{E_{sb}\,{\mit\Psi}_{b\,0}}{(-E_{ss})}\,\frac{1}{1+{\rm e}^{\,{\rm i}\,5\pi/8}\,\lambda\,Q_E^{5/4}}.
\end{equation}
The contribution from the pole at $p=p_+$ is 
\begin{equation}
-\frac{4}{5}\,\frac{E_{sb}\,{\mit\Psi}_{b\,0}}{(-E_{ss})}\,\frac{p_+\,{\rm e}^{\,(p_+-{\rm i\,Q}_E)\,\hat{t}}}{p_+-{\rm i}\,Q_E}.
\end{equation}
Likewise, the contribution from the pole at $p=p_-$ is 
\begin{equation}
-\frac{4}{5}\,\frac{E_{sb}\,{\mit\Psi}_{b\,0}}{(-E_{ss})}\,\frac{p_-\,{\rm e}^{\,(p_--{\rm i\,Q}_E)\,\hat{t}}}{p_--{\rm i}\,Q_E}.
\end{equation}
Finally, the contribution from the branch cut is 
\begin{equation}
\frac{E_{sb}\,{\mit\Psi}_{b\,0}}{(-E_{ss})}\,\frac{\lambda}{\sqrt{2}\,\pi}\int_0^\infty
\frac{{\rm e}^{-(u+{\rm i}\,Q_E)\,\hat{t}}\,u^{5/4}\,du}{(u+{\rm i}\,Q_E)\left(1-\sqrt{2}\,\lambda\,u^{5/4}+\lambda^2\,u^{5/2}\right)}.
\end{equation}
Thus,
\begin{align}
{\mit\Psi}_0(\hat{t}) &=\frac{E_{sb}\,{\mit\Psi}_{b\,0}}{(-E_{ss})}\left\{\frac{1}{1+{\rm e}^{\,{\rm i}\,5\pi/8}\,\lambda\,Q_E^{5/4}}
-\frac{4}{5}\left[\frac{p_+\,{\rm e}^{\,(p_+-{\rm i\,Q}_E)\,\hat{t}}}{p_+-{\rm i}\,Q_E}+\frac{p_-\,{\rm e}^{\,(p_--{\rm i\,Q}_E)\,\hat{t}}}{p_--{\rm i}\,Q_E}\right]\right.\nonumber\\[0.5ex]
&\left.\phantom{=}+\frac{\lambda}{\sqrt{2}\,\pi}\int_0^\infty
\frac{{\rm e}^{-(u+{\rm i}\,Q_E)\,\hat{t}}\,u^{5/4}\,du}{(u+{\rm i}\,Q_E)\left(1-\sqrt{2}\,\lambda\,u^{5/4}+\lambda^2\,u^{5/2}\right)}\right\}.
\end{align}

\subsection{Numerical Calculation}
Let 
\begin{equation}
\frac{E_{sb}\,\bar{\mit\Psi}_{b\,0}}{(-E_{ss})} = 1.
\end{equation}
Let 
\begin{equation}
\frac{S^{1/3}}{(-E_{ss})} ={\mit\Sigma}.
\end{equation}
Thus,
\begin{equation}
{\mit\Psi}_0(\hat{t}) = \frac{1}{2\pi}\int_{-\infty}^{\infty}
\bar{\mit\Psi}_0(g)\,{\rm e}^{\,g\,\hat{t}}\,dx,
\end{equation}
where $g=\sigma + {\rm i}\,x$,  $\sigma$ is real and positive, and 
\begin{equation}
\bar{\mit\Psi}_0(g)= \frac{F_s(g)}{g\left[{\mit\Sigma}\,\hat{\mit\Delta}_s(g)+ 1\right]}.
\end{equation}
Here,
\begin{equation}
F_s(g) = -\frac{1}{8}\sum_{n=0,\infty}\frac{(1/2)_n\,(n+1)_1}{(a)_{2+n}\,2^n},
\end{equation}
where
\begin{align}
a&= \frac{1}{4}\,(Q^{3/2}-1),\\[0.5ex]
Q &= g+ {\rm i}\,Q_E,\\[0.5ex]
(z)_n &= \frac{\Gamma(z+n)}{\Gamma(z)}.
\end{align}
In fact,
\begin{align}
(z)_0 &=1,\\[0.5ex]
(z)_n &= (z+n-1)\,(z)_{n-1}
\end{align}
for $n>0$. 
Furthermore,
\begin{equation}
\hat{\mit\Delta}_s(g) = -\frac{\pi}{8}\,\frac{Q^{5/4}}{a+1/2}\,\frac{\Gamma(a)}{\Gamma(a+1/2)}.
\end{equation}

\section{Solution of Simplified Layer Equations}
\subsection{Simplified Layer Equation}
The simplified  layer equation is
\begin{equation}\label{layer}
\frac{d}{dp}\!\left(\frac{p^2}{Q+p^2}\,\frac{d\bar{\phi}}{dp}\right)-Q\,p^2\,\bar{\phi}=0,
\end{equation}
which can also be written
\begin{align}\label{layer1}
\frac{d^2\bar{\phi}}{dp^2} +\frac{2\,Q}{p\,(Q+p^2)}\,\frac{d\bar{\phi}}{dp} - Q\,(Q+p^2)\,\bar{\phi}=0.
\end{align}
The boundary conditions are $\bar{\phi}(p)\rightarrow 0$ as $p\rightarrow \infty$, and
\begin{equation}
\bar{\phi}(p) = \frac{\hat{\mit\Delta}_s}{\pi\,p} + 1 + {\cal O}(p)
\end{equation}
as $p\rightarrow 0$. 

Let
\begin{equation}
\chi(p) = \frac{p^2}{Q+p^2}\,\frac{d\bar{\phi}}{dp}.
\end{equation}
It follows that
\begin{equation}\label{chi}
\frac{d\chi}{dp}=Q\,p^2\,\bar{\phi}.
\end{equation}
It is also easily seen that $\chi(p)$ satisfies
\begin{equation}
\frac{d}{dp}\left(\frac{1}{p^2}\,\frac{d\chi}{dp}\right)-\frac{Q\,(Q+p^2)}{p^2}\,\chi=0,
\end{equation}
which can also be written
\begin{equation}\label{layera}
\frac{d^2\chi}{dp^2}-\frac{2}{p}\,\frac{d\chi}{dp} - Q\,(Q+p^2)\,\chi = 0.
\end{equation}

\subsection{Small-$p$ Limit}
Let
\begin{equation}
\bar{\phi}(p)= \frac{\hat{\mit\Delta}_s}{\pi\,p} + 1 + a\,p+{\cal O}(p^2).
\end{equation}
Substituting into the layer equation, (\ref{layer1}), we get
\begin{equation}\label{e57}
\bar{\phi}(p) =  \frac{\hat{\mit\Delta}_s}{\pi\,p} + 1 + \frac{\hat{\mit\Delta}_s}{\pi}\left(\frac{Q^2}{2}-\frac{1}{Q}\right)p + {\cal O}(p^2).
\end{equation}

Substituting into the alternative layer equation, (\ref{layera}), and also making use of Eqs.~(\ref{chi}) and (\ref{e57}), we obtain
\begin{equation}\label{chis}
\chi(p) = -\frac{\hat{{\mit\Delta}}_s}{Q\,\pi} + \frac{\hat{{\mit\Delta}}_s\,Q}{2\pi}\,p^2+\frac{Q}{3}\,p^3+{\cal O}(p^4).
\end{equation}

\subsection{Large-$p$ Limit}
Let us write
\begin{equation}
\bar{\phi}(p) = A\,\exp\left(-\frac{Q^{1/2}\,p^2}{2}\right)f(p).
\end{equation}
At large $p$, the layer equation, (\ref{layer1}), yields
\begin{equation}
2\,p\,\frac{df}{dp} +(1+Q^{3/2})\,f = 0.
\end{equation}
The solution is 
\begin{equation}
f(p) = p^{-\alpha},
\end{equation}
where
\begin{equation}
\alpha = \frac{1}{2}\,(1+Q^{3/2}).
\end{equation}
It follows that
\begin{align}\label{lphi}
\bar{\phi}(p)= A\,p^{-\alpha}\,\exp\left(-\frac{Q^{1/2}\,p^2}{2}\right).
\end{align}

Let us write
\begin{equation}
\chi(p) = B\,\exp\left(-\frac{Q^{1/2}\,p^2}{2}\right)g(p).
\end{equation}
At large $p$, the alternative layer equation, (\ref{layera}), yields
\begin{equation}
2\,p\,\frac{dg}{dp} +(-1+Q^{3/2})\,g = 0.
\end{equation}
The solution is
\begin{equation}
g(p)= p^{\beta},
\end{equation}
where
\begin{equation}
\beta = \frac{1}{2}\,(1-Q^{3/2}).
\end{equation}
It follows that
\begin{align}\label{e68}
\chi(p)= B\,p^{\beta}\,\exp\left(-\frac{Q^{1/2}\,p^2}{2}\right).
\end{align}

In order to be in the large-$p$ limit, we require
\begin{align}
p^2\gg |Q|,\\[0.5ex]
|Q|^{1/2}\,p^2\gg 1,
\end{align}
or
\begin{equation}
p\gg |Q|^{1/2}, |Q|^{-1/4},
\end{equation}
or
\begin{equation}
p \gg \frac{(1+|Q|^{3/2})^{1/2}}{|Q|^{1/4}}.
\end{equation}

\subsection{Riccati Transformation}
Let
\begin{equation}
W(p) = \frac{p}{\bar{\phi}}\,\frac{d\bar{\phi}}{dp}.
\end{equation}
It follows from Eq.~(\ref{e57}) that 
\begin{equation}
W(p) = -1 + \frac{\pi\,p}{\hat{\mit\Delta}_s}+{\cal O}(p^2)
\end{equation} 
at small $p$. Furthermore, Eq.~(\ref{lphi}) implies that 
\begin{equation}\label{e63}
W(p) = -Q^{1/2}\,p^2-\alpha
\end{equation}
at large $p$. Substituting into the layer equation, (\ref{layer1}), we deduce that the differential equation that governs $W(p)$ is
\begin{equation}\label{e72}
\frac{dW}{dp} = -\frac{1}{p}\left(\frac{Q-p^2}{Q+p^2}\right)W - \frac{W^2}{p} + Q\,(Q+p^2)\,p.
\end{equation}
The equation is integrated from large $p$, subject to the boundary condition (\ref{e63}), to small $p$. At small $p$, 
\begin{equation}\label{e73}
\hat{\mit\Delta}_s = \frac{\pi}{dW/dp}.
\end{equation}

Let
\begin{equation}
V(p) = \frac{p}{\chi}\,\frac{d\chi}{dp}.
\end{equation}
Substituting into the alternative layer equation, (\ref{layera}), we deduce that the differential equation that governs $V(p)$ is 
\begin{equation}\label{e75}
\frac{dV}{dp} = \frac{3\,V}{p} - \frac{V^2}{p}+Q\,(Q+p^2)\,p.
\end{equation}
According to Eq.~(\ref{chis}), 
\begin{equation}
V(p)= -Q^2\,p^2 - \frac{Q^2\,\pi}{\hat{\mit\Delta}_s}\,p^3
\end{equation}
at small $p$. Moreover, according to Eq.~(\ref{e68}), 
\begin{equation}\label{e77}
V(p) = -Q^{1/2}\,p^2+\beta
\end{equation}
at large $p$. 

\subsection{Plan of Action}
\subsubsection{Stage 1}
Launch solutions of Eqs.~(\ref{e72}) and (\ref{e75}) from large $p$, subject to the respective boundary conditions (\ref{e63}) and (\ref{e77}), and
integrate to small $p$. Save $V(p)$ onto a grid. Deduce the value of $\hat{\mit\Delta}_s(g)$ from Eq.~(\ref{e73}). 

\subsubsection{Stage 2}
Launch the following system of equations from small $p$, 
\begin{align}
\frac{dU}{dp} &= \frac{V(p)}{p},\\[0.5ex]
\frac{dF}{dp} &= \exp[U(p)],
\end{align}
subject to the boundary conditions
\begin{align}
U(0) &=0,\\[0.5ex]
F(0) &= 0,
\end{align}
and integrate to large $p$. Here, $V(p)$ is interpolated from the grid. 
Then
\begin{equation}
F_s(g) = \frac{\hat{\mit\Delta}_s}{Q\,\pi}\,F(\infty).
\end{equation}

\subsection{Stage 3}
Inverse Laplace transform:
\begin{equation}
{\mit\Psi}_0(\hat{t}) = \frac{1}{2\pi}\int_{-\infty}^{\infty}
\frac{F_s(\sigma+{\rm i}\,\omega)\,{\rm e}^{\,(\sigma+{\rm i}\,\omega)\,\hat{t}}}{(\sigma+{\rm i}\,\omega)\left[{\mit\Sigma}\,\hat{\mit\Delta}_s(\sigma+{\rm i}\,\omega)+1\right]}\,d\omega.
\end{equation}

\section{Solution of Full Layer Equation}
\subsection{Full Layer Equation}
The full layer equation is 
\begin{equation}\label{e83}
\frac{d}{dp}\!\left(A\,\frac{dY_e}{dp}\right) - \frac{B}{C}\,p^2\,Y_e,
\end{equation}
where $Y_e(p)$ is the Fourier-Laplace transformed electron stream-function, and
\begin{align}
A &= \frac{p^2}{g + {\rm i}\,(Q_E+Q_e) + p^2},\\[0.5ex]
B&=  (g+{\rm i}\,Q_E)[g+{\rm i}\,(Q_E+Q_i)] + [g+{\rm i}\,(Q_E+Q_i)]\,(P_\varphi+P_\perp)\,p^2+ P_\varphi\,P_\perp\,p^4,\\[0.5ex]
C&= g + {\rm i}\,(Q_E+Q_e) +\{P_\perp + [g+{\rm i}\,(Q_E+Q_i)\,D^2]\}\,p^2 + \iota_e^{-1}\,P_\varphi\,D^2\,p^4.
\end{align}
The boundary conditions are $Y_e\rightarrow 0$ as $p\rightarrow\infty$, and
\begin{equation}
Y_e(p) = \frac{\hat{\mit\Delta}_s}{\pi\,p}+ 1+{\cal O}(p)
\end{equation}
as $p\rightarrow 0$. We also need
\begin{equation}
F_s(g) = -\int_0^\infty A\,\frac{dY_e}{dp}\,dp.
\end{equation}

Let 
\begin{equation}\label{e89}
\chi(p) = A\,\frac{dY_e}{dp}.
\end{equation}
It follows from Eqs.~(\ref{e83}) and the previous two equations that
\begin{equation}\label{e90}
A\,\frac{d}{dp}\!\left(\frac{C}{B\,p^2}\,\frac{d\chi}{dp}\right) - \chi = 0,
\end{equation}
as well as
\begin{equation}
F_s= -\int_0^\infty \chi(p)\,dp.
\end{equation}

\subsection{Small-$p$ Limit}
Let
\begin{equation}
Y_e(p) = \frac{\hat{\mit\Delta}_s}{\pi\,p} + 1+ a\,p+b\,p^2+{\cal O}(p^3).
\end{equation}
Substituting into the generalized layer equation, (\ref{e83}), we get
\begin{align}
Y_e(p) &= \frac{\hat{\mit\Delta}_s}{\pi\,p} + 1
\nonumber\\[0.5ex]
&\phantom{=}+\frac{\hat{\mit\Delta}_s}{\pi}\left\{
\frac{1}{2}\,(g+{\rm i}\,Q_E)\,[g+{\rm i}\,(Q_E+Q_i)]-\frac{1}{g+{\rm i}\,(Q_E+Q_e)}\right\}p\nonumber\\[0.5ex]
&\phantom{=} + \frac{1}{6}\,(g+{\rm i}\,Q_E)\,[g+{\rm i}\,(Q_E+Q_i)]\,p^2+{\cal O}(p^{\,3}).
\end{align}
Substituting into Eq.~(\ref{e89}), we get
\begin{align}
\chi(p) &= -\frac{\hat{\mit\Delta}_s}{[g+{\rm i}\,(Q_E+Q_e)]\,\pi}+\frac{{\mit\Delta}_s}{2\pi}\,\frac{(g+{\rm i}\,Q_E)\,[g+{\rm i}\,(Q_E+Q_i)]}{g+{\rm i}\,(Q_E+Q_e)}\,p^2\nonumber\\[0.5ex]
&\phantom{=} + \frac{1}{3} \frac{(g+{\rm i}\,Q_E)\,[g+{\rm i}\,(Q_E+Q_i)]}{g+{\rm i}\,(Q_E+Q_e)}\,p^3+{\cal O}(p^4).
\end{align}

\subsection{Large-$p$ Limit}
In the large-$p$ limit, if we write
\begin{align}
A &=1+\frac{\alpha}{p^2},\\[0.5ex]
\frac{B}{C} &= \beta+\frac{\gamma}{p^2},
\end{align}
and
look for a solution of Eq.~(\ref{e89}) of the form
\begin{equation}
Y_e(p) \propto p^x\,\exp\left(\frac{-\sqrt{\beta}\,p^2}{2}\right)
\end{equation}
then we find that
\begin{equation}
x = \frac{\gamma -\sqrt{\beta}\,(1-\sqrt{\beta}\,\alpha)}{2\sqrt{\beta}}.
\end{equation}
It is easily seen that
\begin{align}
\alpha &= -[g+ {\rm i}\,(Q_E+Q_e)],\\[0.5ex]
\beta&= \frac{\iota_e\,P_\perp}{D^2},\\[0.5ex]
\gamma&= \frac{\iota_e\,P_\perp}{D^2}\left(1+ [g+{\rm i}\,(Q_E+Q_i)]\,\frac{P_\varphi+P_\perp}{P_\varphi\,P_\perp}\right.\nonumber\\[0.5ex]
&\phantom{=}\left.-\{P_\perp+[g+{\rm i}\,(Q_E+Q_i)\,D^2]\}\,\frac{\iota_e}{P_\varphi\,D^2}\right).
\end{align}

Finally, it is easily seen from Eq.~(\ref{e89}) that
\begin{equation}
\chi(p) \propto p^{x+1}\,\exp\left(\frac{-\sqrt{\beta}\,p^2}{2}\right)
\end{equation}
at large-$p$. 

In order to be in the large-$p$ limit, we require
\begin{align}
p&\gg |g+{\rm i}\,(Q_E+Q_e)|^{1/2},\\[0.5ex]
p&\gg\left|\frac{[g+{\rm i}\,(Q_E+Q_i)]\,(P_\varphi + P_\perp)}{P_\varphi\,P_\perp}\right|^{1/2},\\[0.5ex]
p&\gg \left|\frac{(g+{\rm i}\,Q_E)\,[g+{\rm i}\,(Q_E+Q_i)]}{P_\varphi\,P_\perp}\right|^{1/4},\\[0.5ex]
p&\gg \left|\frac{(P_\perp+[g+{\rm i}\,(Q_E+Q_i)\,D^2])}{\iota_e^{-1}\,P_\varphi\,D^2}\right|^{1/2},\\[0.5ex]
p&\gg \left|\frac{g+{\rm i}\,(Q_E+Q_e)}{\iota_e^{-1}\,P_\varphi\,D^2}\right|^{1/4},\\[0.5ex]
p&\gg \left(\frac{\iota_e^{-1}\,P_\varphi\,D^2}{P_\varphi\,P_\perp}\right)^{1/4}.
\end{align} 

\subsection{Low-$D$ Limit}\label{lowd}
If
\begin{align}
D^2\ll \left|\frac{P_\perp}{\hat{g} + {\rm i}\,(Q_E+Q_e)}\right|, ~\frac{\iota_e\,P_\perp}{P_\varphi^{2/3}}
\end{align}
then the terms in the layer equations involving $D^2$ are negligible. In this case, in the large-$p$ limit,  we can write
\begin{align}
A&= 1+\frac{\alpha}{p^2},\\[0.5ex]
\frac{B}{C} &= \beta\,p^2+\gamma,
\end{align}
where
\begin{align}
\alpha &= - [\hat{g}+{\rm i}\,(Q_E+Q_e)],\\[0.5ex]
\beta &= P_\varphi,\\[0.5ex]
\gamma&= -{\rm i}\,(Q_e-Q_i)\,\frac{P_\varphi}{P_\perp} + \hat{g} + {\rm i}\,(Q_E+Q_i).
\end{align}
The solution of Eq.~(\ref{e89}) becomes
\begin{equation}
\hat{Y}_e\propto p^{-1}\exp\left(x\,p - \frac{\sqrt{\beta}\,p^3}{3}\right)
\end{equation}
where
\begin{equation}
x = \frac{\alpha\,\beta-\gamma}{2\sqrt{\beta}}.
\end{equation}

Finally, it is easily seen from Eq.~(\ref{e89}) that
\begin{equation}
\chi(p) \propto  p\exp\left(x\,p - \frac{\sqrt{\beta}\,p^3}{3}\right),
\end{equation}
at large-$p$. 

In order to be in the large-$p$ limit, we require
\begin{align}
p&\gg |\hat{g}+{\rm i}\,(Q_E+Q_e)|^{1/2},\\[0.5ex]
p&\gg\left|\frac{[\hat{g}+{\rm i}\,(Q_E+Q_i)]\,(P_\varphi + P_\perp)}{P_\varphi\,P_\perp}\right|^{1/2},\\[0.5ex]
p&\gg \left|\frac{(\hat{g}+{\rm i}\,Q_E)\,[\hat{g}+{\rm i}\,(Q_E+Q_i)]}{P_\varphi\,P_\perp}\right|^{1/4},\\[0.5ex]
p&\gg \left|\frac{\hat{g}+{\rm i}\,(Q_E+Q_e)}{P_\perp}\right|^{1/2},\\[0.5ex]
p&\gg P_\varphi^{-1/6}.
\end{align} 

\subsection{Ricatti Transformation}
Let 
\begin{equation}
W= \frac{p}{Y_e}\,\frac{dY_e}{dp}.
\end{equation}
The generalized layer equation, (\ref{e83}), transforms to give
\begin{equation}\label{e72a}
\frac{dW}{dp} =- \frac{A'}{p}\,W -\frac{W^{\,2}}{p} + \frac{B}{A\,C}\,p^3,
\end{equation}
where
\begin{equation}
A' = \frac{g+{\rm i}\,(Q_E+Q_e)-p^2}{g+{\rm i}\,(Q_E+Q_e)+p^2}.
\end{equation}
This equation must be solved subject to the boundary condition that
\begin{equation}\label{e63a}
W(p) = x-\sqrt{\beta}\,p^2
\end{equation}
at large-$p$, and
\begin{equation}\label{e73a}
W (p)=-1+\frac{\pi\,p}{\hat{\mit\Delta}_s}
\end{equation}
at small-$p$. 
However, in the low-$D$ limit,
\begin{equation}\label{e63aa}
W(p) = -1 +x\,p-\sqrt{\beta}\,p^3
\end{equation}
at large-$p$. 

Let 
\begin{equation}
V= \frac{p}{\chi}\,\frac{d\chi}{dp}.
\end{equation}
Equation~(\ref{e90}) transforms to give 
\begin{equation}\label{e75a}
\frac{dV}{dp} = 2\,p\,(B'-C')\,V + \frac{3\,V}{p} - \frac{V^2}{p} + \frac{B}{A\,C}\,p^3,
\end{equation}
where 
\begin{align}
B' &= \frac{ [g+{\rm i}\,(Q_E+Q_i)]\,(P_\varphi+P_\perp)+ 2\,P_\varphi\,P_\perp\,p^2}{ (g+{\rm i}\,Q_E)[g+{\rm i}\,(Q_E+Q_i)] + [g+{\rm i}\,(Q_E+Q_i)]\,(P_\varphi+P_\perp)\,p^2+ P_\varphi\,P_\perp\,p^4},\\[0.5ex]
C'&=\frac{\{P_\perp + [g+{\rm i}\,(Q_E+Q_i)\,D^2\}+ 2\,\iota_e^{-1}\,P_\varphi\,D^2\,p^2}{g + {\rm i}\,(Q_E+Q_e) +\{P_\perp + [g+{\rm i}\,(Q_E+Q_i)\,D^2\}\,p^2 + \iota_e^{-1}\,P_\varphi\,D^2\,p^4}.
\end{align}
This equation must be solved subject to the boundary condition that
\begin{equation}\label{e77a}
V(p) = 1+x-\sqrt{\beta}\,p^2
\end{equation}
at large-$p$. 
However, in the low-$D$ limit,
\begin{equation}\label{e63aa}
V(p) = 1 +x\,p-\sqrt{\beta}\,p^3
\end{equation}
at large-$p$. 


\subsection{Revised Plan of Action}
\subsubsection{Stage 1}
Launch solutions of Eqs.~(\ref{e72a}) and (\ref{e75a}) from large $p$, subject to the respective boundary conditions (\ref{e63a}) and (\ref{e77a}), and
integrate to small $p$. Save $V(p)$ onto a grid. Deduce the value of $\hat{\mit\Delta}_s(g)$ from Eq.~(\ref{e73a}). 

\subsubsection{Stage 2}
Launch the following system of equations from small $p$, 
\begin{align}
\frac{dU}{dp} &= \frac{V(p)}{p},\\[0.5ex]
\frac{dF}{dp} &=\exp[U(p)],
\end{align}
subject to the boundary conditions
\begin{align}
U(0) &=0,\\[0.5ex]
F(0) &= 0,
\end{align}
and integrate to large $p$. Here, $V(p)$ is interpolated from the grid. 
Then
\begin{equation}
F_s(g) = \frac{\hat{\mit\Delta}_s}{[g+{\rm i}\,(Q_E+Q_e)]\,\pi}\, F(\infty).
\end{equation}

\subsection{Stage 3}
Inverse Laplace transform:
\begin{equation}
{\mit\Psi}_0(\hat{t}) = \frac{1}{2\pi}\int_{-\infty}^{\infty}
\frac{F_s(\sigma+{\rm i}\,\omega)\,{\rm e}^{\,(\sigma+{\rm i}\,\omega)\,\hat{t}}}{(\sigma+{\rm i}\,\omega)\left[{\mit\Sigma}\,\hat{\mit\Delta}_s(\sigma+{\rm i}\,\omega)+1\right]}\,d\omega.
\end{equation}

\end{document}
